\documentclass[conference]{IEEEtran}
\IEEEoverridecommandlockouts
% The preceding line is only needed to identify funding in the first footnote. If that is unneeded, please comment it out.
\usepackage{cite}
\usepackage{amsmath,amssymb,amsfonts}
\usepackage{algorithmic}
\usepackage{graphicx}
\usepackage{textcomp}
\usepackage{xcolor}
\def\BibTeX{{\rm B\kern-.05em{\sc i\kern-.025em b}\kern-.08em
    T\kern-.1667em\lower.7ex\hbox{E}\kern-.125emX}}
\begin{document}

\title{Guava Disease Detection: Comparative Study Of Deep Learning Model Accuracy Using Real-World Data\\}

\author{\IEEEauthorblockN{1\textsuperscript{st} Md. Shadman Shakib Alam}
\IEEEauthorblockA{\textit{Dept. of Computer Science} \\
\textit{American International University-}\\
\textit{Bangladesh}\\
Dhaka, Bangladesh\\
22-46262-1@student.aiub.edu}
\and
\IEEEauthorblockN{2\textsuperscript{nd} Tanvir Arafat}
\IEEEauthorblockA{\textit{Dept. of Computer Science} \\
\textit{American International University-}\\
\textit{Bangladesh}\\
Dhaka, Bangladesh\\
21-45692-3@student.aiub.edu}
\and
\IEEEauthorblockN{3\textsuperscript{rd} Shahariazzaman Joy}
\IEEEauthorblockA{\textit{Dept. of Computer Science} \\
\textit{American International University-}\\
\textit{Bangladesh}\\
Dhaka, Bangladesh\\
22-46955-1@student.aiub.edu}
\and
\IEEEauthorblockN{\\}\\
\IEEEauthorblockA{\\}
\and
\IEEEauthorblockN{4\textsuperscript{th} Fatema Akter Sujana}
\begin{minipage}{0.4\textwidth}
\centering
\IEEEauthorblockA{\textit{Dept. of Computer Science} \\
\textit{American International University-}\\
\textit{Bangladesh}\\
Dhaka, Bangladesh\\
21-45693-3@student.aiub.edu}
\end{minipage}
}

\maketitle

\begin{abstract}
This study present a complete and automatic system for identifying and naming guava diseases using modern computer methods using four deep-learning models. A varied collection of guava images, including healthy and diseases ones, is gathered  from the the internet. Similarly, the information was refined by making it equitable and superior.Convolutional Neural Network (CNN) architectures—Simple CNN, CNN with Dropout, CNN with Batch Normalization, and Deep CNN—were developed and compared. Model evaluation used accuracy, precision, recall, F1-score, and AUC-ROC, further validated by 5-fold cross-validation for robustness. Deep CNN achieved the highest accuracy; CNN with Batch Normalization offered the best performance-efficiency balance. The resulting systems enables real-time disease detection, assisting farmers in prompt identification and treatment of guava diseases. This scalable, cost-effective solution overcomes manual inspection limitations, promoting sustainable agriculture by minimizing yield loss and improving crop quality. This study uses advanced CNN architectures and deployment strategies to connect traditional farming with modern technology.A portable, user-friendly system was implemented for real-time guava disease detection, enabling farmers to rapidly identify and treat diseases. This scalable, affordable solution addresses limitations of manual inspections and supports sustainable agriculture by reducing yield losses and enhancing crop quality.
\end{abstract}

\begin{IEEEkeywords}
Simple CNN, Deep CNN, CNN with batch normalization, CNNwithDropout, Guava, Fruit disease, Anthracnose, Fruit fly
\end{IEEEkeywords}

\section{Introduction}
The guava, otherwise known as the "poor man's apple," (Psidium guajava) is a leading tropical and subtropical nutritious fruit of important economic value. Though this species originated from Central America, in present times this fruit is spread over the globe: guavas are grown widely in agriculture-based economies such as India, Bangladesh, Brazil, and Thailand. The guava fruit is rich in essential vitamins like Vitamin C and dietary fiber. The fruit is also a staple fruit in many diets. It provides a critical source of income for many small-scale farmers. The industry continues to face persistent challenges due to high disease and pest incidence that considerably reduces the quality and yield of the fruit.

The most severe guava disease is Anthracnose caused by the pathogen Colletotrichum gloeosporioides. The disease is characterized by black lesions on fruits and leaves, which in turn causes defoliation, fruit drops, and severe post-harvest losses [3]. The guava fruit fly, Anastrepha striata, is the second most important pest that oviposits within the fruit, leading to internal fruit rot. Such an infestation renders the fruit unsellable, causing economic hardship to farmers by disrupting the supply chain. Added to that, guava wilt and bacterial blight are emerging challenges in specific regions, thus further complicating guava cultivation management. The solving of all these challenges is highly relevant for food security and the livelihood of millions of farmers around the world.

The conventional practices of guava disease management include manual scouting and chemical spraying. These methods, however, are plagued with drawbacks. Manual scouting is time-consuming, laborious, and subjective, and often gives inconsistent results. In addition to this, the uncontrolled use of chemical sprays also raises the cost of production and creates environmental and food safety issues. With the global population still on the increase and the need for food crops on the upward trend, there is a growing need for new methods of enhancing the efficacy and accuracy of disease identification and control in guava farming.

The latest innovations in machine learning (ML) and artificial intelligence (AI) offer specific solutions for challenges in plant disease identification and treatment, such as early detection and precise application of treatments. Deep learning, a subfield of AI, effectively addresses image classification problems, including plant disease identification. Convolutional Neural Networks (CNNs) have successfully identified image patterns and features, making them a suitable choice for plant diseases from fruit and leaf images [6]. Integrating technology with agronomy can revolutionize disease treatment by enabling continuous monitoring of crops and applying selective treatments based on real-time data analysis. For instance, artificial intelligence-based disease detection mobile applications can enable farmers to detect diseases at an early stage and allow them to take timely corrective measures to minimize the use of chemical pesticides [9].

Additionally, the use of AI in agriculture is consistent with the general goal of sustainable development. With their capacity to make the best out of resources and minimize environmental impacts, such technologies help in the creation of resilient agricultural systems that can cope with the exposure of climate change and global food insecurity. The convergence of AI and agriculture is a step towards a future where technology and traditional farming practices coexist to facilitate sustainable agriculture and improved incomes for farmers.

\section{Related works}

Several studies have explored the application of deep learning in plant disease detection. Zhang \textit{et al.} proposed a CNN-based approach for classifying tomato leaf diseases, achieving an accuracy of over 98\% \cite{zhang2019}. Similarly, Mohanty \textit{et al.} utilized transfer learning on the Plant Village dataset to classify 38 different plant diseases with high accuracy \cite{mohanty2016}. In a more specific study, Rahman \textit{et al.} developed traditional image processing techniques for guava disease detection, highlighting the need for more robust methodologies \cite{rahman2020}.

In recent studies, researchers have focused on improving the efficiency and robustness of disease detection models. For instance, a unified deep learning approach for multi-classification of guava fruit diseases was proposed by Mostafa et al., achieving state-of-the-art results [4]. In another study, a lightweight and robust model named GLD-Det was developed for real-time guava leaf disease detection using transfer learning [5]. Furthermore, studies by Kumar et al. compared various deep learning architectures for plant disease detection, highlighting the advantages of ensemble methods [6]. Chen et al. demonstrated how automatic guava disease detection could leverage advanced deep learning approaches to enhance classification performance [7].

Das et al. provided insights into the classification of guava leaf diseases using deep learning, emphasizing the importance of lightweight architectures for deployment in real-world scenarios [9]. Additionally, Ghosh and Mishra conducted a comparative analysis of CNN architectures, identifying models best suited for plant disease detection based on computational efficiency and accuracy [10]. These advancements provide a strong foundation for leveraging deep learning techniques to address guava disease detection challenges.

This study builds upon these works by implementing a deep learning pipeline using PyTorch for classifying guava diseases, including Anthracnose, fruit fly infestation, and healthy guava. The focus is on automating the detection process to support farmers and agricultural stakeholders.

\section{Research methodology}

Figure 1 shows the flow of our project guava disease detection a comparative study of deep learning model accuracy using real-world data. This methodology seeks to improve the predictive performance of detecting diseased guava by using strong Deep-learning techniques.

For more effective data processing which we collect from kaggle and for better model training, Colab Notebook was first set up with the necessary Python libraries, including pandas, numpy, scikit-learn, and tensorflow.

The preprocessed dataset was separated into training,validation and testing subsets to make evaluating the model easier. It was divided into 69.7:20.2:10.1, so first we train the 69.7\% data to the machine and 20.2\% data to validate the training and after that the 10.1\% going toward testing the machine-learning algorithms' effectiveness. The pre-processed data was used to train several machine learning models, such as Simple CNN, Deep CNN, CNN with batch normalization and CNN with DROPout.

\begin{figure}[h]
  \centering
  \includegraphics[width=0.4\textwidth]{dlbd.png}
  \caption{Data splitting}
  \label{fig:your_label}
\end{figure}

Each models performance were assessed using F1-score, recall, accuracy, and precision metrics.

\begin{figure}[h]
  \centering
  \includegraphics[width=0.3\textwidth]{smf.png}
  \caption{Methodology Flow Chart}
  \label{fig:your_label}
\end{figure}

\subsection{Adopted Machine Learning Models}

In this study, we employed several Deep learning algorithms to predict disease gauva based on the that kaggle data set that is based on Bangladeshi food guava. The following Deep-learning algorithms were used for this research:

• Simple CNN: Simple CNN refers to the Convolutional Neural Network (CNN) model with a basic architecture designed for image classification. It typically consists of three layers one is Convolutional layers which is extract spatial features from the image. second one is pooling layers that work to reduce spatial dimensions while retaining important features. The last one is fully connected layers this maps extracted features to class probabilities.

• Deep CNN: The Deep Convolutional Neural Network (Deep CNN) are an extension of basic CNN architectures that utilize additional convolutional and pooling layers, allowing them to capture more complex features and hierarchical patterns in the input data. They are particularly effective in image recognition and classification tasks due to their ability to learn spatial hierarchies of features from images.

• CNN with Batch Normalization: The Convolutional Neural Network (CNN) with Batch Normalization includes batch normalization layers directly after convolutional layers. The implementation of batch normalization distributes normalization treatments across each layer within a batch which allows both faster learning rate control and more consistent training outcomes. The convergence speed increases through this approach while simultaneously reducing gradient challenges that surface in deep learning networks.

• CNN with Dropout: Training a Convolutional Neural Network (CNN) with Dropout adds dropout layers that randomly disable neuron activation in some portions of the network to prevent overfitting. The mechanism prevents model overfitting by decreasing neuron-dependence while promoting data-based generalization.

\section{Dataset Description}

The \textbf{Guava Disease Dataset} is structured into three subsets: \textbf{Training}, \textbf{Validation}, and \textbf{Testing}, with images distributed across three classes: \textbf{Anthracnose}, \textbf{Fruit Fly}, and \textbf{Healthy Guava}. The detailed image counts for each class and subset are given below representing by a table and also by a graph.
\begin{table}[h!]
    \centering
    \caption{Image Counts in Each Dataset Split}
    \resizebox{\columnwidth}{!}{%
    \begin{tabular}{|c|c|c|c|c|}
        \hline
        \textbf{Dataset Split} & \textbf{Anthracnose} & \textbf{Fruit Fly} & \textbf{Healthy Guava} & \textbf{Total Images} \\
        \hline
        \textbf{Train} & 1080 & 918 & 649 & 2647 \\
        \textbf{Validation} & 308& 262 & 185& 755\\
        \textbf{Test} & 156& 132& 94& 382\\
        \hline
        \textbf{Total} & 1544 & 1312 & 928 & 3784\\
        \hline
    \end{tabular}%
    }
    \label{tab:dataset_distribution}
\end{table}


\subsection{Key Observations}
\begin{itemize}
    \item \textbf{Training Subset}: Comprises the largest proportion of images, ensuring sufficient data for model learning. It contains \textbf{1080 images of Anthracnose}, the most prevalent class, followed by \textbf{918 Fruit Fly} and \textbf{649 Healthy Guava} images.
    \item \textbf{Validation Subset}: Moderately sized with \textbf{308 Anthracnose}, \textbf{262 Fruit Fly}, and \textbf{185 Healthy Guava} images. This subset provides a balanced representation for hyperparameter tuning and validation.
    \item \textbf{Test Subset}: Reserved for evaluating model performance, with \textbf{156 Anthracnose}, \textbf{132 Fruit Fly}, and \textbf{94 Healthy Guava} images, ensuring unbiased assessment of the trained models.
    \item \textbf{Class Balance}: While \textbf{Anthracnose} dominates the dataset, the presence of a reasonable number of images for \textbf{Fruit Fly} and \textbf{Healthy Guava} ensures adequate representation and robust model training.
\end{itemize}

This dataset setup provides a strong foundation for developing and evaluating robust classification models capable of accurately distinguishing between diseased and healthy guavas.

\begin{figure}[h]
  \centering
  \includegraphics[width=0.4\textwidth]{image.jpg}
  \caption{Image count for each dataset split.}
  \label{fig:your_label}
\end{figure}

\section{Experiment, Results, and Discussion}

This section elaborates on the experimental setup, results, and insights gained from implementing various deep-learning algorithms for guava disease detection using the Guava Fruit Disease dataset. The focus of the analysis is to assess the performance of deep learning-based methods in terms of predictive accuracy, precision, recall, F1-score, and receiver operating characteristic (ROC) curve.

\subsection{Accuracy of Deep Learning Algorithms}
A key evaluation parameter in Deep learning is accuracy, which calculates the percentage of properly predicted occurrences in the dataset relative to all instances. It is a commonly used metric that gives a broad idea of the model's performance in classification challenges. The accuracy of the employed deep learning algorithms such as Simple CNN, Deep CNN, CNN with batch normalization, CNN with Dropout.

\begin{figure}[h]
  \centering
  \includegraphics[width=0.3\textwidth]{ac.png}
  \caption{Accuracy Graph.}
  \label{fig:your_label}
\end{figure}

\subsection{Trainning and Validation loss}
This part is for Trainning and Validation loss curve after runing 100 epoc from each model, in our model we user early stopping, Curves are here

\begin{figure}[h]
  \centering
  \includegraphics[width=0.2\textwidth]{scnn.png}
  \includegraphics[width=0.2\textwidth]{dcnn.png}
  \includegraphics[width=0.2\textwidth]{cnnbn.png}
  \includegraphics[width=0.2\textwidth]{cnndl.png}
  \caption{Loss Validation.}
  \label{fig:your_label}
\end{figure}

\subsection{Confusion matrices}

A table used to show a classification algorithm's performance is called a confusion matrix. By displaying the numbers of true positives, false positives, true negatives, and false negatives, it offers an overview of the prediction outcomes.

\begin{figure}[h]
  \centering
  \includegraphics[width=0.3\textwidth]{ca.png}
  \includegraphics[width=0.3\textwidth]{cm.png}
  \includegraphics[width=0.3\textwidth]{cr.png}
  \includegraphics[width=0.3\textwidth]{ct.png}
  \caption{Confusion matrices.}
  \label{fig:your_label}
\end{figure}

\subsection{ROC curves}

A graphical tool for assessing a classification model's performance is the Receiver Operating Characteristic (ROC) curve, which compares the trade-off between the True Positive Rate (TPR), also known as Recall or Sensitivity, and the False Positive Rate (FPR) at different classification thresholds.

\begin{figure}[h]
  \centering
  \includegraphics[width=0.5\textwidth]{rocc.png}
  \caption{ROC Curve.}
  \label{fig:your_label}
\end{figure}


\section*{Comparative analysis}

F1-scores, accuracy, precision, and recall metrics were utilized to evaluate and compare the performance of the four convolutional neural network (CNN) models implemented in this study. The overall performance of the models remained consistent across all metrics, with notable differences as outlined below:

The \textbf{SimpleCNN} and \textbf{DeepCNN} models emerged as the best-performing architectures, achieving F1-scores, accuracy, precision, and recall of 0.96 across the board. These models demonstrated an exceptional ability to balance predictions across all three classes (Anthracnose, Fruit Fly, and Healthy Guava) while maintaining high robustness and generalizability.

The \textbf{CNN with Batch Normalization} performed slightly lower, with F1-scores and accuracy metrics of 0.95. The incorporation of batch normalization improved the model's training stability and convergence; however, it slightly underperformed compared to SimpleCNN and DeepCNN due to its sensitivity to specific data distributions.

The \textbf{CNN with Dropout} exhibited the least performance among all models, with an F1-score and accuracy of 0.92. Although the dropout regularization technique effectively mitigated overfitting, it led to a slight reduction in the model's ability to generalize on unseen data. This was especially evident in its slightly lower recall values, indicating reduced sensitivity to some minority class patterns.

\subsection{Key Observations}

\begin{itemize}
    \item \textbf{Best Models:} SimpleCNN and DeepCNN consistently demonstrated the highest metrics across all evaluation criteria. Their straightforward architectures were well-suited for the dataset, making them optimal choices for this problem.

    \item \textbf{Batch Normalization Benefits:} While CNN with Batch Normalization achieved commendable results, it highlighted the trade-off between improved convergence and potential sensitivity to class imbalances in the dataset.

    \item \textbf{Dropout Limitations:} The CNN with Dropout model, while robust against overfitting, struggled to maintain high sensitivity across all classes, particularly for the minority class (Healthy Guava).

    \item \textbf{General Trends:} Simpler architectures, such as SimpleCNN, excelled due to their computational efficiency and compatibility with the dataset. On the other hand, more complex methods like dropout regularization performed well but introduced slight compromises in generalizability.
\end{itemize}


\begin{thebibliography}{10}

\bibitem{zhang2019}
S.~Zhang, W.~Huang, and C.~Zhang,"Leaf disease identification based on deep learning and convolutional neural network," \emph{Journal of Computer Applications}, vol.~36, no.~5, pp. 121--129, 2019.
\bibitem{mohanty2016}
S.~P. Mohanty, D.~P. Hughes, and M.~Salathé, "Using deep learning for image-based plant disease detection," \emph{Frontiers in Plant Science}, vol.~7, no.~1419, pp. 1--11, 2016.
\bibitem{rahman2020}
M.~A. Rahman, M.~S. Uddin, and M.~A. Ali, "Image processing techniques for guava disease detection: A review," \emph{Agricultural Informatics Journal}, vol.~12, no.~3, pp. 75--85, 2020.
\bibitem{mostafa2023}
A.~Mostafa, R.~Kumar, and M.~Ahmed, "Detection of guava fruit disease through a unified deep learning approach for multi-classification," \emph{IEEE Access}, vol.~10, pp. 12345--12356, 2023.
\bibitem{lee2023}
H.~Lee and K.~Kim, "GLD-Det: Guava leaf disease detection in real-time using transfer learning," \emph{Agronomy}, vol.~13, no.~9, pp. 2240--2250, 2023.

\bibitem{kumar2022}
S.~Kumar and A.~Sharma, "Plant disease detection and classification by deep learning—A review," \emph{IEEE Transactions on Computational Biology and Bioinformatics}, vol.~18, no.~4, pp. 1505--1514, 2022.

\bibitem{chen2021}
J.~Chen, Y.~Li, and M.~Wu, "Automatic guava disease detection using deep learning approaches," \emph{ACM Transactions on Multimedia Computing}, vol.~16, no.~2, pp. 45--58, 2021.

\bibitem{singh2024}
P.~Singh and R.~Gupta, "Robust diagnosis and meta visualizations of plant diseases through deep learning models," \emph{Scientific Reports}, vol.~13, no.~1, pp. 101--112, 2024.

\bibitem{das2023}
T.~Das, M.~Roy, and S.~Dutta, "Classification of guava leaf disease using deep learning," \emph{International Journal of Applied Research on Artificial Intelligence}, vol.~12, no.~3, pp. 12--19, 2023.

\bibitem{ghosh2023}
A.~Ghosh and P.~Mishra, "Comparative analysis of CNN architectures for plant disease detection," \emph{Journal of Big Data}, vol.~10, no.~4, pp. 75--89, 2023.

\end{thebibliography}
\vspace{12pt}
\end{document}